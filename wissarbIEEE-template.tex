%%%%%%%%%%%%%%%%%%%%%%%%%%%%%%%%%%%%%%%%%%%%%%%%%%%%%%%%%%%%%%%%%%%%%%%%%%%%%%%%
%2345678901234567890123456789012345678901234567890123456789012345678901234567890
%        1         2         3         4         5         6         7         8

%\documentclass[letterpaper, 10 pt, conference]{orbieeeconfpre}  % Comment this line out if you need a4paper

\documentclass[a4paper, 10pt, journal]{wissarbIEEE}      % Use this line for a4 paper
%\conference{IEEE Conference for Awesome ORB Research}

\bibliographystyle{orbref-num}

\overrideIEEEmargins                                      % Needed to meet printer requirements.

% See the \addtolength command later in the file to balance the column lengths
% on the last page of the document

\usepackage{hyperref}
\usepackage{graphicx}
\usepackage{tabularx}
\usepackage{booktabs}
\usepackage{lipsum}

\title{\LARGE \bf Scientific Work Short Paper \\Feasibility Study - SmartWarehouse}

\author{Felix Hausberger and Robin Kuck}

\begin{document}

\maketitle

%%%%%%%%%%%%%%%%%%%%%%%%%%%%%%%%%%%%%%%%%%%%%%%%%%%%%%%%%%%%%%%%%%%%%%%%%%%%%%%%
\begin{abstract}

In this short paper the object detectors \textit{You Only Look Once} and \textit{Single Shot MultiBox Detector} are compared for precision, reactivity, training and inference behaviour and examined for their potential for industrial use. The background scenario of the Smart Warehouse offers live video data of a drone with goods in a warehouse, which are to be classffied and localized in real time. In the future, this should make it possible to carry out inventories and inventory analyses of a warehouse in a time- and cost-efficient manner conserving resources.

The goal of this feasibility study is to find out whether the Smart Warehouse scenario is technically feasible. In addition, the focus is also on the object detectors themselves, their differences in architecture, behavior and how well they are generally suitable for industrial application scenarios.

\end{abstract}

%%%%%%%%%%%%%%%%%%%%%%%%%%%%%%%%%%%%%%%%%%%%%%%%%%%%%%%%%%%%%%%%%%%%%%%%%%%%%%%%
\section{Introduction}

Object detection represents a major field of study in industrial fields like autonomous driving, industrial processing or even government monitoring. Especially in times of the industrial change towards Industry 4.0 such object detectors represent an optimization potential not to be neglected, e.g. in warehousing and logistics. Combined with an autonomous drone, such object detectors could make it possible to conduct inventory checks in a warehouse without human assistance. How different object detectors behave when applied to a real time industry scenario like \textit{SmartWarehouse} should be evaluated in this short paper. As being a feasibility study, the main goal of this work also is to discuss the feasibility of the \textit{SmartWarehouse} idea based on the performance of the two selected object detectors \textit{You Only Look Once} (YOLO) and \textit{Single Shot MultiBox Detector} (SSD) in terms of precision, responsiveness, training and inference behavior. In \autoref{relatedwork} related work to similar industry scenarios should be introduced before explaining the approach and architecture of the \textit{SmartWarehouse} prototype in \autoref{architecture}. In \autoref{evaluation} the results of the feasibility study will be presented and evaluated before the interpretability of the results is discussed in \autoref{results}. \autoref{conclusion} gives a quick coonclusion about the main achievements of this short paper.

\section{Related Work} \label{relatedwork}

The idea of automized inventory checks with drones is not new. \cite{doks.innovationGmbH.2020} uses a similar approach, but uses normal RFID technology or simple barcodes to identify a product. Using this approach only a small number of instances at a time can be identified, while using object detection algorithms enable many-numbered and faster processing. 

The YOLO algorithm was introduced in \cite{JosephRedmonSantoshDivvalaRossGirshickAliFarhadi.2016} and uses ....

Paper \cite{WeiLiuDragomirAnguelovDumitruErhanChristianSzegedyScottReedChengYangFuAlexander.2016} introduced the SSD algorithm, which ....

\section{Architecture of the SmartWarehouse scenario} \label{architecture}

\section{Results and evaluation} \label{evaluation}

\section{Discussion} \label{results}

\section{Conclusion} \label{conclusion}

\bibliography{mybibfile}

\end{document}
